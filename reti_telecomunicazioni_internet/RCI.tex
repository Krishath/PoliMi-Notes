%% LyX 2.1.4 created this file.  For more info, see http://www.lyx.org/.
%% Do not edit unless you really know what you are doing.
\documentclass[english]{article}
\usepackage[T1]{fontenc}
\usepackage[latin9]{inputenc}
\usepackage{fancyhdr}
\pagestyle{fancy}

\makeatletter

%%%%%%%%%%%%%%%%%%%%%%%%%%%%%% LyX specific LaTeX commands.
%% Because html converters don't know tabularnewline
\providecommand{\tabularnewline}{\\}

\makeatother

\usepackage{babel}
\begin{document}

\title{Reti di Telecomunicazione ed Internet}


\author{Giulio De Pasquale}

\maketitle

\section{Legenda}
\begin{description}
\item [{Tempo~di~trasmissione}] $T\,[ms];$
\item [{Tempo~di~propagazione}] $\tau\,[ms];$
\item [{Finestra~di~pacchetti}] $W\,[pkts];$
\item [{Probabilit�~di~successo~di~trasmissione}] $P\,(0\leq P\leq1);$
\item [{Capacit�~trasmissiva~di~un~canale}] $C\,[\frac{bit}{s}]$;
\item [{Grandezza~di~un~pacchetto}] $p\,[bit];$
\item [{Sequence~Number}] $SN;$
\item [{Request~Number}] $RN$;
\end{description}

\section{Data Link Layer}


\subsection{Controllo d'errore}


\subsubsection{Stop and Wait}

Il mittente invia un messaggio e attende dal destinatario una conferma
positiva (ACK), negativa (NACK) o un comando; se scade il time-out
per uno di questi tre, il mittente provveder� a rispedire il pacchetto
e il destinatario si incaricher� di scartare eventuali repliche. Nel
caso in cui si verificasse un errore nella trasmissione del segnale
di conferma, il mittente provveder� a rinviare il pacchetto; il destinatario
ricever� in questo modo una copia del pacchetto gi� ricevuto, credendo
che gli sia pervenuto un nuovo pacchetto. Per ovviare a questo problema
si pu� procedere numerando i pacchetti trasmessi, ovvero inserendo
un bit di conteggio.
\begin{description}
\item [{Efficienza}] $\eta=\frac{T}{T+2\tau}$;
\end{description}

\subsubsection{Go-Back-N}

Il mittente dispone di un buffer dove immagazzina $W$ pacchetti da
spedire, man mano che riceve la conferma ACK svuota il buffer e lo
riempie con nuovi pacchetti; nell'eventualit� di pacchetti persi o
danneggiati e scartati avviene il reinvio del blocco di pacchetti
interessati. I pacchetti ricevuti dal destinatario dopo quello scartato
vengono eliminati.
\begin{description}
\item [{Finestra~minima}] $W\geq1+\frac{2T}{\tau};$
\item [{Tempo~di~trasmissione}] $T=\frac{2\tau}{W-1};$
\item [{Efficienza~(con~errori)}] $\frac{P}{(1-P)(1+\frac{2\tau}{T})+P};$
\end{description}

\subsubsection{Selective Repeat}

Il mittente reinvia i pacchetti oltre il timeout ed un numero di frame
specificato dalla grandezza della finestra anche senza attendere l'arrivo
di ACK individuali da parte del ricevitore; egli pu� rifiutare un
pacchetto singolarmente che pu� essere ritrasmesso indipendentemente
ed inoltre accetta pacchetti non ordinati inserendoli in un buffer.


\subsection{HDLC}

Si tratta di un protocollo a riempimento di bit e usa la tecnica del
bit stuffing per evitare che le sequenze di terminazione compaiano
all'interno dei frame.

I frame dati HDLC possono essere trasmessi attraverso collegamenti
sincroni o asincroni. In essi per� non esiste alcuna delimitazione,
quindi, per ognuno di essi, viene usato una flag / delimitatore ($'01111110'$,
in esadecimale '$0x7E$') che non potr� mai apparire nel resto del
frame. Quando non sta venendo trasmesso nessun frame, il canale viene
occupato da sole flag.

Il Frame Check Sequence (FCS) � un CRC-CCITT 16 bit oppure CRC-32
a 32 bit calcolato sui campi Indirizzo, Controllo e Dati. Se i pacchetti
hanno un FCS integro, viene spedito un pacchetto di conferma (ACK)
a chi aveva trasmesso in modo da permettergli di spedire il prossimo
frame. Altrimenti il ricevente manda una conferma negativa (NACK)
o, pi� semplicemente, scarta il frame. Se manda il NACK ed esso arriva
a chi aveva trasmesso, pu� essere spedito un altro frame.


\subsubsection{Tipi di stazioni e trasmissione dati}

I tipi di stazioni sono 3:
\begin{itemize}
\item Terminale \emph{primario}: � responsabile delle operazioni di controllo
sul collegamento. Manda i frame di controllo (comandi). 
\item Terminale \emph{secondario}: lavora sotto il controllo di quello primario.
Spedisce solo pacchetti di risposta. Il primario � collegato ai secondari
attraverso collegamenti logici multipli. 
\item Terminale \emph{combinato}: ha le caratteristiche di entrambi i terminali
sopra. Spedisce sia comandi sia risposte. 
\end{itemize}
I dati invece, possono essere trasmessi nelle seguenti modalit�:
\begin{itemize}
\item \emph{NRM (Normal Response Mode)}: in cui un terminale primario inizia
a trasmettere e il secondario risponde se interpellato. Questo permette
la comunicazione su canali half-duplex.
\item \emph{ARM (Asynchronous Response Mode)}: aggiunto successivamente
per permettere l'utilizzo su canali full-duplex � fondamentalmente
come l'NRM, con la differenza che un terminale secondario pu� trasmettere
anche senza l'autorizzazione di un terminale primario.
\item \emph{ABM (Asynchronous Balanced Mode)}: in cui interagiscono terminali
combinati.
\end{itemize}

\subsubsection{Tipologie di frame HDLC}

Esistono tre principali tipologie di frame:
\begin{itemize}
\item \emph{Information Frames}, oppure I-frames: trasportano dati dal livello
di rete. In pi� possono anche includere informazioni sul controllo
flusso dati e d'errore tramite piggybacking. 
\item \emph{Supervisory Frames}, oppure S-frames: sono usati per il controllo
del flusso dati e d'errore quando il piggybacking � impossibile o
di scarsa efficienza. I frame S non hanno il campo Dati. 
\item \emph{Unnumbered Frames}, oppure U-frames: sono utilizzati per diversi
scopi, tra cui la gestione del collegamento. Questi frame non sempre
hanno il campo Dati.
\end{itemize}
\begin{table}[b]
\caption{Struttura di un frame HDLC}


\begin{tabular}{|c|c|c|c|c|c|}
\hline 
Flag & Indirizzo & Controllo & Dati & FCS & Flag\tabularnewline
\hline 
\hline 
8 bit & 8 bit o pi� & 8 bit / 16 bit & Variabile (anche nulla) & 16 bit o 32 bit & 8 bit\tabularnewline
\hline 
\end{tabular}
\end{table}


\begin{table}[b]
\caption{Campo Controllo dei frame HDLC}


\begin{tabular}{|c|c|c|c|c|c|c|c|c|}
\hline 
 & 0 & 1 & 2 & 3 & 4 & 5 & 6 & 7\tabularnewline
\hline 
\hline 
I-Frame & 0 & \multicolumn{3}{c|}{SN} & P/F & \multicolumn{3}{c|}{RN}\tabularnewline
\hline 
S-Frame & 1 & 0 & \multicolumn{2}{c|}{tipo} & P/F & \multicolumn{3}{c|}{RN}\tabularnewline
\hline 
U-Frame & 1 & 1 & \multicolumn{2}{c|}{tipo} & P/F & \multicolumn{3}{c|}{tipo}\tabularnewline
\hline 
\end{tabular}
\end{table}



\subsubsection{Il bit P/F}

Poll / Final � un bit con due nomi. E' Poll quando un terminale primario
richiede un'informazione da un terminale secondario mentre � Final
quando il terminale secondario fornisce la risposta a quello primario
oppure quanto la trasmissione termina.

Il bit viene utilizzato come token tra tutti i terminali e pu� esisterne
solo uno per volta. Il terminale secondario invia un Final solo quando
riceve un Poll da un terminale primario; alla stessa maniera, un terminale
primario reinvia un Poll solo quando riceve un Final da un terminale
secondario oppure al termine di un timeout, indicando che il bit �
stato perso.


\subsubsection{I bit tipo}

I due bit tipo si trovano nelle trame S ed in quelle U.


\paragraph{S-Frame}
\begin{description}
\item [{Receiver~Not~Ready~(RNR)}] Conferma la ricezione dei pacchetti
ricevuti sino a quel momento ($RN-1$) e richiede di interromperne
ulteriori invii sino ad un'ulteriore comunicazione.
\item [{Receiver~Ready~(RR)}] E' utilizzato idealmente come un ACK. Indica
che il mittente � pronto per ricevere ulteriori dati, annulla l'effetto
di un precedente pacchetto $RNR$ e conferma la ricezione dei pacchetti
ricevuti sino a quel momento ($RN-1$).
\item [{Reject~(REJ)}] Richiede l'immediata ritrasmissione dei pacchetti
a partire da $SN$ e conferma la ricezione dei pacchetti ricevuti
sino a quel momento ($RN-1$). Utilizzato quando non viene rispettato
l'ordine dei pacchetti.
\item [{Selective~Reject~(SREJ)}] E' utilizzato idealmente come un NACK.
Richiede la singola ritrasmissione del pacchetto $SN$.
\end{description}

\subsection{Controllo di Flusso}
\begin{description}
\item [{Rate~massimo~di~trasmissione}] $R=min(C,\frac{W\cdot p}{2\tau+T})$;
\item [{Durata~di~trasferimento~attivo~(passivo)}] $T_{ON}\,(T_{OFF})\,[s]$;
\item [{Frequenza~media~di~trasmissione}] $A=\frac{p}{T_{ON}+T_{OFF}}$;
\end{description}

\subsubsection{Burstiness $B$}

La ``burstiness'' � l'intermittenza di incremento e decremento di
attivit� o frequenza di un dato evento; per calcolarla ci avvaliamo
dei tempi di incremento $T_{ON}$ e decremento $T_{OFF}$.

$B=\frac{T_{ON}}{T_{ON}+T_{OFF}}$; $(0\leq B\leq1)$
\end{document}
